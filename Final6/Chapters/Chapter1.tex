% Chapter 1

\chapter{Introduction} % Write in your own chapter title
\label{Chapter1}
\lhead{Chapter 1. \emph{Introduction}} % Write in your own chapter title to set the page header
Tunnelling spectroscopy of superconductors is a heavily studied topic. The typical experimental results are modelled  based on the work of Blonder,Tinkham, and Klapwijck(BTK)\citep{Reference6} who proposed a theory focusing on the tunnelling spectroscopy between a normal metal and a conventional $s$-wave superconductor . Extension of the theory, however, should be explored as in BTK, the calculation of tunnelling spectroscopy is for simple $s$-wave case. Also, the assumed step function of pair potential should be questioned. In addition, there are newly conducted experiments, showing some new features of the tunnelling spectroscopy\citep{Reference7,Reference8,PhysRevLett.77.3025}. 

One extension of the BTK theory discussed by various authors\citep{Reference2,Reference9,PhysRevLett.74.3451} is to establish theory of the d-wave tunnelling spectroscopy, which is required by various experiments\citep{PhysRevB.78.092505,Reference3}. Another more advanced approach is to include the proximity effect, where we need to numerically solve the Bogoliubov equations due to the increased complexity of having a spatially dependent superconducting gap\citep{Reference4,Reference5,Reference8}. Previous theoretical studies of the superconducting proximity effect ,however, were limited primarily to s-wave superconductors, as it was believed to be exceedingly difficult to generate a proximity effect with the high temperature superconductors. Nonetheless this long held belief was recently shown to be incorrect by our groups newly invented mechanical bonding technique. To get a better understanding of our experimental results in our group, it is necessary to establish a d-wave tunnelling
spectroscopy theory accounting for proximity efect, which hasn�t been clearly discussed
before. This thesis discusses our recent effort to approach the simulation of this effect by establishing the model of proximity tunnelling spectroscopy.

The general approach always follows the following steps. First, we write the Hamiltonian for the material, from which we derive the function of the pair potential. Second, we solve the Bogoliubov equations with assumptions and the obtained pair potential, from which we get the so called tunnelling conductance kernel for a specific solid angle. Third, the integration over the half-fermi sphere is conducted so that we get the total tunnelling conductance. Forth, the total tunnelling conductance joins the convolution with the differential fermi distribution function. With the previous model established, we apply some additional algorithms like genetic algorithm to fit the experimental data. 


